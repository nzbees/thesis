%*******************************************************
% Abstract
%*******************************************************
%\renewcommand{\abstractname}{Abstract}
\pdfbookmark[1]{Abstract}{Abstract}
\begingroup
\let\clearpage\relax
\let\cleardoublepage\relax
\let\cleardoublepage\relax

\chapter*{Abstract}

New Zealand has around thirty different species of \emph{native bees} and much has been discovered about their biology. They are pollinators of wild and cultivated plants and are likely important to the health of ecosystems. Most are solitary ground nesting bees, commonly referred to as mining bees because individual females construct their nests in the ground. During the active flight season, many thousands of individuals nest alongside each other to form large communities called \emph{aggregations}.

However, studies of native bees can be difficult so there is much to learn about their diversity and population status. To address this problem, a method to measure populations of native bees using digital images and semi-automated image analysis is proposed and capitalises on unique aspects of their nesting biology. While it is difficult to acquire images of individual bees it is  comparatively straightforward to photograph nests. Furthermore, the number of nests in an aggregation can provide a good indicator of community health. For this reason methods centred on counting the number of active nests which are used as a proxy for populations.
 
Image data were collected over four years from three communities of native bees located in Whangarei (Northland, New Zealand). Fundamental ecological data were collected including  manual nest counts. Biomedical imaging platform Fiji was used to process all images. Pixel-level segmentation was implemented via the \emph{Random Forest} algorithm. The  performance of Random Forest classifiers were analysed in the workbench Weka. Manual nest estimates taken in the field, and from images were compared to semi-automated nest counts to verify the automated imaging method. 

Combined with the ease of model training, construction and application, the Random Forest classifier was well suited to the imaging task. The classifier compared favourably with similar types of machine-learners in analysis performed with Weka. The automated count results correlated well with the manual nest counts taken in the field and manual nest counts taken from images; all methods indicate the number of active nests at the locations monitored have changed over fours years.
 
The imaging methodology presented in this thesis shows good potential to help increase base-line knowledge and understanding of the population status of native bees in New Zealand. It can also be rapidly adapted for other solitary ground nesting bees worldwide and help to provide much needed information on the health of other important background pollinators. 

\vfill

\endgroup

\vfill


