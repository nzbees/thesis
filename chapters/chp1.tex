\chapter{Introduction}\label{ch:introduction}
Pollination is a natural process pivotal to life on earth. Any changes in services has the potential to affect food security and human welfare \cite{Kearns1998}. The importance of {pollinators} was highlighted by the popular press in 2006 when {declines} in populations of honey bees, \emph{Apis mellifera} (Hymenoptera: Apoidea) were documented throughout the United States (US). The term \emph{Colony Collapse Disorder} was coined to describe the syndrome which rapidly travelled the globe causing alarm. Not surprisingly, as around 80-85\% of all cultivated food crops are dependent on pollination from insects; bees are a valuable pollinator group \cite{Allsopp2008}. Studies estimate the economic cost of pollination services could be as high as \$310.9 million US dollars \cite{Allsopp2008}. Others indicate the economic consequences of pollinator declines could result in a reduction of food crops amounting to a total of \$334.1 billion US dollars \cite{Bauer2010}.

The pollination crisis continues to receive attention throughout Europe and the US although the causes of mass honey bee losses remain largely unknown \cite{Williams2010}. An added consequence of this phenomenon has been to increase the overall awareness of the role of pollinators. Especially the role of non-\emph{Apis} bees. Many species have been undervalued in the past \cite{Buchmann1997} but in the future may provide a buffer against major colony loss events \cite{NewstromL2013}. In the US, native bees are estimated to contribute about \$3 billion dollars towards fruit production each year \cite{Losey2006}. O'Toole \cite{Otoole2002} explains, "...native bees occupy keystone positions and without them, ecosystems would eventually collapse..." \cite[pg.32]{Otoole2002}

Around 20,000 different bees have been formally described by science \cite{Michener2000}. Most of these species are solitary \cite{Michener2000}. Unlike social bees, solitary species do not produce honey or live in colonies. They are not as easily managed for crop pollination. One exception is the alkali bee, \emph{Nomia midlander} (Hymenoptera: Halictidae). It is the most intensively managed solitary bee in the world and is a vital pollinator for alfalfa crops. Alkali bees have a tendency to nest alongside each other so many thousands of bees can form large communities. Some of these communities can persist over decades \cite{Canej2003}. One of the most populous and long-lived nesting sites was recorded by Cane \cite{Cane2008}. He conducted a monitoring programme on alkali bees over eight years by measuring their populations across 240 km$^2$ of agricultural land (Washington, US). 

There are at least 40 different species of bees in New Zealand. Around 32 of these are native bees \cite{Donovan2007}. Most of New Zealand's native bees are solitary ground nesting types. Many have nesting behaviours which are similar to the alkali bee \cite{Canej2003}. They can be described as \emph{gregarious} ground nesting bees. Therefore, at the beginning of the active flight season each year (around September-October), female native bees start to construct their nests in the ground. They prefer to nest alongside each other. Consequently within a very short period of time large communities or aggregations are formed by thousands of nesting bees.

In contrast to alkali bees, New Zealand's native bees have not been managed for crop pollination. Baseline knowledge about some species is limited. There are also challenges associated with studies of native bees. Thus long term monitoring programmes are also difficult to manage. Since most species are in flight for only a few months of the year, projects can be drawn out and laborious. Good population data can take longer than three years to collect \cite{Cane2008}. As a result it can be difficult to  attract researchers and funding. Most native bees are hard to identify with the naked eye because they look the same. Expert training is required in order to identify them properly and even with the aid of taxonomic keys identification can be difficult \cite{Wheeler2004,de2007}. Many native species are small. Therefore when they are in mid-flight they are nearly impossible to visually track. Capturing digital images of native bees in mid-flight is equally problematic. Tracking tools that have been successfully used on larger insects cannot be easily adapted for New Zealand's native bees \cite{Osborne1999, Hart2007}. Looking towards global research, habitat fragmentation (from urbanisation and agricultural intensification) is an important issue \cite{Steffan2003}. In the future, communities of native bees may be increasingly more difficult to locate and study.

There are relatively few past or present studies on New Zealand's native bees. Historical research was more often focused on taxonomy, biology and floral relationships. One of the earliest studies was by Rayment \cite{Rayment1935}. He described the life history of a M\={a}ori bee; presenting a range of sketches of larvae, pupae and adult bees. Kelly \cite{Kelly1996} captured   video footage of native bees prising open the flowers of an endangered species of mistletoe. The mistletoe has bird-adapted flowers and was previously thought to be only visited by birds.
Donovan's \emph{Apoidea} \cite{Donovan2007} is the most comprehensive body of research; it is a complete taxonomic treatment of New Zealand's bees. Donovan \cite{Donovan2010} has also made progress towards relocating native bees. He tested methods for establishing new nests using a species of native bee, \emph{Leioproctus huakiwi} (Hymenoptera: Colletidae). His results were promising. Hart \cite{Hart2004,Hart2007} reviewed different methods for studying the foraging ranges of native bees. She collated natural history records from communities in and around Whangarei (Northland, New Zealand) over a three year period. The results from her study showed that at one community up to seven different species of native bees were nesting alongside each other. The diversity of species located within the communities of native bees around Mt. Parihaka (Whangarei) were the highest that have ever been recorded in New Zealand. Newstrom-Lloyd \cite{NewstromL2013} has provided a comprehensive and current account of pollination services in New Zealand. She suggested more could be done to explore the role of native bees as managed pollinators in New Zealand. 

One native bee species (\emph{Leioproctus nunui} (Hymenoptera: Colletidae) is critically endangered, according to a recent report \cite{Ward2012}. However, it is difficult to ascertain the conservation status of other species. Native bees might provide essential pollination services. If so, their role New Zealand's ecosystems could be critical. Studies from abroad indicate the consequences of losing pollinator species can be unpredictable and irreversible \cite{Kevan2003}. Some studies have shown the ecological effects might only be  recognised when dependent species of wild and cultivated crops decline \cite{Kevan2003}. Although the research to date is promising \cite{Hart2007, Donovan2010, NewstromL2013} much more is required. Long term monitoring initiatives would help to quantify the population abundance and species diversity of New Zealand's native bees. Image-centric tools could also be used, simplifying ecological methods. If the lessons from abroad are considered relevant, monitoring initiatives might even be community driven. Therefore this thesis outlines the development, design, and application of an image-centric monitoring tool for New Zealand's native ground nesting bees.

\section{Objectives}\label{sec:objectives}
The main aim of this research was to design, apply and verify the performance of an image-centric monitoring system. The system was designed to measure the population abundance of ground nesting native bees in New Zealand. The techniques for monitoring bees were trialled during the field studies in 2009. The types of digital data were not limited to images. Audio and video formats were also considered relevant. In 2010 field tests were completed. There was sufficient proof of concept to formulate a clear image-centric design and direction for applied field monitoring. The research thus centred on digital image acquisition and analysis of whole communities of native bees. Five main questions were used to refine the research objectives:

\begin{enumerate}
\item What indicators could be used to establish the general health of bee communities?
\item Could images be used to capture/quantify \emph{key} indicators?
\item Can image handling, acquisition and analysis be standardised?
\item What pattern recognition/classification techniques best suit the image data?
\item What methods can be used to verify the accuracy/precision of imaging methods?
\item The research questions listed above were used to identify five key aims. The objectives were therefore to:
\begin{itemize}
\item Investigate the types of image data suitable for proxy measures.
\item Develop standard image collection techniques.
\item Design manual field sampling method for comparative analysis against the image method.
\item Apply the methods proposed: gather manual and image data in field tests, conducted over multiple years and locations.
\item Select data handling protocols including: storage, collation, processing and analysis.
\item Verify the accuracy and precision of imaging methods.
\item Compare manual field and imaging methods; and results.
\end{itemize}
\end{enumerate}

The key objectives formed the basis of several recursive tasks. These were categorised into stages as shown in Table \ref{tab:rd} below.

\begin{table}[!htbp]{myfloatalign}\caption[Development stages]{Development stages}\label{tab:rd} 
\begin{tabular}{p{1.2in}p{3.1in}}\toprule
Stage & Tasks \\ \midrule
\emph{Review Data} & Examine past image data and ascertain design constraints; and identify appropriate monitoring sites. \\
\emph{Image Acquisition} & Test techniques for standard image acquisition. \\
\emph{Field Methods} & Physically survey locations and select sites for repeat monitoring.\\ & Design field methods for comparative analysis against imaging methods.\\ & Collect manual and image field data over multiple seasons. \\ 
\emph{Image Processing} & Collate and prepare images. \\ & Train and apply classifiers to segment target objects in images. \\ &
Post--process and count key data. \\
\emph{Verification} & Verify the precision and accuracy of image classifiers. \\ 
\emph{Analysis} & Examine and compare manual field and imaging results; review final monitoring data. \\ \bottomrule
\end{tabular}
\end{table}

\subsection{Thesis outline}\label{sec:thesis-outline}
\begin{flushright}
\textbf{Part I: Literature Review and Theory}\\
\end{flushright}
There were two main literature reviews. They were draw from the scientific and technology dimensions of this research. Both were relevant to the development and design of the image-centric monitoring system outlined. The first review was dedicated to examining the field methods used for studying native bees or closely related species. The second review was dedicated to examining the types of imaging methods that are used in biology and life sciences.\\
\\
\emph{Chapter 2 Monitoring Native Bees}\\
\\
Chapter 2 starts with a broad overview of the challenges associated with studies of native bees. The field methods designed for a scientific analysis of native bees or their habitats, are generally very difficult to conduct. There are considerable practical and logistical problems with the methods that are currently used to estimate the population or diversity status of native bees. These issues probably had a cumulative impact; they restricted both the quality and quantity of studies that could be, and were reviewed for this research. Therefore a wider literature search was necessary. Studies with similar types of insects or broadly related research methods were sought. The relevance of literature was constantly referenced back to the specific research tasks outlined. 

A broader review of current literature was conducted to show some tools were more practical than others;  a wider perspective also revealed there were some studies that were more relevant than others. Especially when a simplified view of the monitoring task was adopted. Regarding the most suitable tools that could aid studies of New Zealand's native bees, the main conclusions that were drawn from a review of current literature settled on digital images and analysis technologies. \\
\\
\emph{Chapter 3 Image Analysis}\\
\\
Biomedical image analysis theory provides the foundation for developing methodologies in a range of other life science disciplines. Software developments within the field of medical image analysis are well represented at the very forefront of research. Significant advances have been achieved, particularly in the areas of open source imaging tools such as ImageJ and \acs{Fiji}. Both are very powerful but intuitive packages and central to the success of many studies.
\\

\begin{flushright}
\textbf{Part II: Methods}
\end{flushright}
\emph{Chapter 4 Field Collection Methods}\\
\\
Chapter 4 details the design of field monitoring methods. The number of active bees nests were used to estimate the populations of native bees. Three different nesting communities located in and around Whangarei (Northland, New Zealand) were monitored over a five year period (2009--2014). Four areas of representative nests were selected at each location. Areas were marked so the same nests could be monitored throughout and across the active bee seasons. Four plastic grids (245 x 245 mm) were placed over the nest areas. The number of active nests inside the grids were manually counted. Digital images of the grids were collected using a standard \ac{DSLR} camera. Each monitoring day, after all three field collections were completed, the digital images were immediately copied from the camera memory \ac{SD} card, onto an external hard-drive for analysis.\\
\\
\emph{Chapter 5 Imaging Methods}\\
\\
Chapter 5 details the imaging tools and methods used to process the images of active nests. This includes the management of the digital data and the design of the image collections database. Image analysis tools and methods are fully detailed so they can be easily replicated for use in similar applications. Open source biomedical image analysis software \ac{Fiji} was used for most of the imaging tasks. The main task for nest monitoring photographs, was delineating the areas in images that corresponded to active nests from other background areas or objects. Therefore this chapter includes the tests used to investigate the performances of a range of image segmentation tools. 

The images of active nests were sufficiently segmented by using an interactive trainable tool in the \ac{TWS} plug-in which is included in the \ac{Fiji} package. \ac{TWS} utilises human knowledge for image segmentations by combining the traces selected by a user which represent key objects. The \ac{TWS} procedures used to classify images of active nests were more dynamic compared to than classical segmentation techniques. Since \ac{TWS} uses an interactive process, replicating the exact nest image segmentations could be problematic. Thus the specific methods used to train, construct and apply monitoring classifiers were fully detailed in this chapter. This included the methods used to test, optimise and verify the performance of the \ac{RF} machine learner, which was the selected classifier used for segmenting monitoring images via \acp{TWS}.
\\
\begin{flushright}
\textbf{Part III: Research Outcomes}
\end{flushright}
\emph{Chapter 6 Results}\\
To do...\\ \\
\emph{Chapter 7 Discussions}\\
To do...\\ \\
\emph{Chapter 8 Conclusions}\\
To do...\\ \\
\subsection{Publications and contributions}\label{sec:publications-and-contributions} 
Publications from this research are listed below. Other outcomes and pathways were also important and revolved around disseminating knowledge to a wider more general audience. It was possible to share some of the research objectives and results with a larger community audience by combining contemporary digital media with M\={a}ori design and narratives. The images collected during field monitoring were integrated with modern designs to create a exhibition; this was hosted by the School of Creative Design and Technologies (AUT).

\begin{enumerate}
\item Hart N. H. and L. Huang. \emph{Monitoring Populations of Solitary Bees using Image Processing Techniques}. International Journal of Computer Applications in Technology, 50(1): 45--50, 2014.- \cite{Hart2014}
\item Hart N. H. and Huang L. \emph{Counting Insects in Fight using Image Processing Techniques}. In Proceedings of the 27th Conference on Image and Vision Computing New Zealand, pages 274--278. ACM, 2012. - \cite{Hart2012}
\item Hart N. H. and Huang L. \emph{Monitoring Nests of Solitary Bees using Image Processing Techniques}. In Mechatronics and Machine Vision in Practice (M2VIP), 2012 19th International Conference, pages 1--4. IEEE, 2012. - \cite{Hart2012a}
\item Hart N. H. and Huang L. \emph{An Image Based Approach to Monitor New Zealand Native Bees}. In Robotics, Automation and Mechatronics (RAM), 2011 IEEE Conference on, pages 353--357. IEEE, 2011. - \cite{Hart2011}
\end{enumerate}


